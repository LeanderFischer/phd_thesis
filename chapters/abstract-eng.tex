The observation of neutrino oscillations has established that neutrinos have non-zero masses. This phenomenon is not explained by the \textit{standard model (SM)} of particle physics, but one viable explanation to this dilemma is the existence of \textit{heavy neutral leptons (HNLs)} in the form of right-handed neutrinos. Depending on their mass and coupling to SM neutrinos, these particles could also play an important role in solving additional unexplained observations such as \textit{dark matter (DM)} and the \textit{baryon asymmetry of the universe (BAU)}. This work presents the first search for HNLs with the IceCube Neutrino Observatory. The standard three flavor neutrino model is extended by adding a fourth GeV-scale mass state and allowing mixing with the tau neutrino through the mixing parameter \ut4. Three HNL mass values, $m_4$, of \SI{0.3}{\gev}, \SI{0.6}{\gev}, and \SI{1.0}{\gev} are tested using ten years of data, collected between 2011 and 2021, resulting in constraints for the mixing parameter of \ut4$ < 0.19\;(m_4 = \SI{0.3}{\gev})$, \ut4$ < 0.36\;(m_4 = \SI{0.6}{\gev})$, and \ut4$ < 0.40\;(m_4 = \SI{1.0}{\gev})$ at \SI{90}{\percent} confidence level. No significant signal of HNLs is observed for any of the tested masses. This first analysis lays the fundamental groundwork for future searches for HNLs in IceCube.
