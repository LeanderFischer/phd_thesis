The observation of neutrino oscillations has established that neutrinos have non-zero masses. This phenomenon is not explained by the standard model of particle physics, but a viable explanation to this dilemma is the existence of heavy neutral leptons, which are right-handed neutrinos with masses much larger then the observerd neutrino masses ($\gg$\si{\electronvolt}).

This work presents the first search for heavy neutral leptons with the IceCube Neutrino Observatory. The standard three flavor neutrino model is extened by adding a fourth heavy mass state and allowing mixing with the tau neutrino through the mixing parameter \ut4. The strength of this mixing is tested using atmospheric neutrinos as a source flux. Muon neutrinos that oscillated into tau neutrinos can produce heavy neutral leptons through neutral current interactions, which then decay back to standard model particles. Both production and decay may produce observable light in the dector, leading to a unique signature of two cascades at low energies.

The measurement is performed through a binned, maximum likelihood fit, comparing the observed data to the expected events from atmospheric neutrinos and heavy neutral leptons. Three mass values $m_4$ of \SI{0.3}{\gev}, \SI{0.6}{\gev}, and \SI{1.0}{\gev} are tested using ten years of data, taken between 2011 and 2021. The fits constrain the mixing parameter to \ut4$ < 0.09\;(m_4 = \SI{0.3}{\gev})$, \ut4$ < 0.21\;(m_4 = \SI{0.6}{\gev})$, and \ut4$ < 0.24\;(m_4 = \SI{1.0}{\gev})$ at $68 \si{\percent}$ confidence level. No significant signal of heavy neutral leptons is observed for any of the tested masses, and the best fit mixing values obtained are consistent with the null hypothesis of no mixing.

Additionally, a thorough investigation of the unique low energy double cascade signature of HNLs in IceCube is performed. A benchmark reconstruction performance is estimated using a well established IceCube reconstruction tool, after optimizing it for low energy double cascade events. The limitations of the detector to observe these events are intentified and their origins are discussed. This lays the fundamental groundwork for future searches for heavy neutral leptons in IceCube.
