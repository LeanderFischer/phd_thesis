The observation of neutrino oscillations has established that neutrinos have non-zero masses. This phenomenon is not explained by the standard model of particle physics, but one viable explanation to this dilemma is the existence of heavy neutral leptons in the form of right-handed neutrinos. Depending on their mass and coupling to standard model neutrinos, these particles could also play an important role in solving additional unexplained observations such as dark matter and the baryon asymmetry of the universe. This work presents the first search for heavy neutral leptons with the IceCube Neutrino Observatory. The standard three flavor neutrino model is extended by adding a fourth GeV-scale mass state and allowing mixing with the tau neutrino through the parameter \ut4. Three heavy neutral lepton mass values, $m_4$, of \SI{0.3}{\gev}, \SI{0.6}{\gev}, and \SI{1.0}{\gev} are tested using ten years of data, collected between 2011 and 2021. No significant signal of heavy neutral leptons is observed for any of the tested masses. The resulting constraints for the mixing parameter are \ut4$ < 0.19\;(m_4 = \SI{0.3}{\gev})$, \ut4$ < 0.36\;(m_4 = \SI{0.6}{\gev})$, and \ut4$ < 0.40\;(m_4 = \SI{1.0}{\gev})$ at \SI{90}{\percent} confidence level. This first analysis lays the fundamental groundwork for future searches for heavy neutral leptons in IceCube.
