\setchapterstyle{kao}
\setchapterpreamble[u]{\margintoc}

\chapter{Detecting Low Energetic Double Cascades}
\labch{double_cascade_performance}


\section{Reconstruction}

\subsection{Table-Based Minimum Likelihood Algorithms}

\subsection{Double Cascade Hypothesis}

\subsection{Modification to Low Energy Events}

\section{Cross Checks}
\subsection{Simplistic Sets}

After generation the events are processed with standard Photon, Detector, L1, and L2 processing and then Taupede+MuMillipede is run on top of the L2 files. Multiple versions with different parameters were produced, some with the OscNext baseline parameters, some without detector noise (in Det level) and some with h2-50cm holeice model, to match the holeice model that was used to generate the photonics tables.


\paragraph{BrightDom Cleaning}

To investigate the effect of the BrightDom cleaning cut the 194601 set without detector noise (and baseline hole ice model) is used. The BrightDom cleaning is needed to stop a few DOMs with many photon hits to drive the reconstruction because this leads to large biases in the energy estimations. Historically, the BrightDom cleaning was removing all DOMs that had a charge larger than 10 times the mean charge. After quickly checking some charge distributions and how the mean behaves it was clear that the cut should better be defined based on a metric that is less affected by outliers, like the median. \reffig{bright_dom_cleaning_charges_mean_median} shows where the mean and the median are located for an example event. The cut was re-defined to use the median instead of the mean and 10\% of the simulation were processed with \href{https://github.com/LeanderFischer/I3_HNL_Decay/blob/a6838ec48e0a2d4f6547cbe064d2928ec55fb76d/submission_scripts/process/process_Taupede.py}{Taupede} using 30x and 100x the median as BrightDom cutoff. \reffig{bright_dom_cleaning_charges_median_scales} shows where these values fall for the same example event.

% \begin{figure}[h!]
%     \subfloat[\labfig{bright_dom_cleaning_charges_mean_median}]{
%         \includegraphics[width=.45\linewidth]{figures/upgoing_string_81_gen_level/brightdom_cleaning/median_and_mean_L2_00001.i3.zst_SplitInIcePulses_frame_0.png}
%         }
%     \subfloat[\labfig{bright_dom_cleaning_charges_median_scales}]{
%         \includegraphics[width=.45\linewidth]{figures/upgoing_string_81_gen_level/brightdom_cleaning/median_scales_L2_00001.i3.zst_SplitInIcePulses_frame_0.png}
%         }
%     \caption{Charge distribution of example event showing mean and median charge (left) and different scales of median charge (right).}
%     \labfig{bright_dom_cleaning_charges}
% \end{figure}

As a quick check of the performance of both cuts the decay length resolution/bias and the resolutions/biases of all energies were checked. The reconstructed decay length is almost not affected by applying this cut, which is as expected, because it is mostly dependent on the arrival time of the photons. The effect on the reconstructed energy is much stronger, where a looser cut (100x) shows a significantly larger bias than the tighter cut at (30x). Even though this was not a highly sophisticated optimization of the BrightDom cut, an improvement was achieved by changing from mean to median and selecting the tighter cut (of the two tested). It's hard to tell how this would perform for high energy events, but I'm quite certain that a definition based on the median would be more reliable than on the mean.

% \begin{figure}[h!]
%     \subfloat[\labfig{bright_dom_cleaning_performance_decay_length_bias}]{
%         \includegraphics[width=.45\linewidth]{figures/upgoing_string_81_gen_level/brightdom_cleaning/median_decay_length_bias_fullset_larger_range_unweighted.png}
%         }
%     \subfloat[\labfig{bright_dom_cleaning_performance_total_energy_bias}]{
%         \includegraphics[width=.45\linewidth]{figures/upgoing_string_81_gen_level/brightdom_cleaning/compare_bright_dom_cuts_fractional_reco_total_energy_error_fully.png}
%         }
%     \caption{Decay length bias (left) and total energy bias (right).}
%     \labfig{bright_dom_cleaning_performance}
% \end{figure}

\section{Performance}

\subsection{Energy/Decay Length Resolution}

\subsection{Double Cascade Classification}
