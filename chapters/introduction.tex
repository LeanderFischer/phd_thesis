\setchapterstyle{kao}
\setchapterpreamble[u]{\margintoc}

\chapter{Introduction}
\labch{intro}

\todo{(Re-)write introduction for PhD thesis (just copy paste from M.Sc.).}

The neutrino was postulated by Wolfgang Pauli \sidecite{Pauli:1930pc} in 1930 to explain the continuous energy spectrum of electrons originating from beta decay.
Cowan and Reines confirmed this prediction of a light, neutral particle in 1956 when they discovered the electron neutrino using inverse beta decay \sidecite{Cowan103}.
Two additional neutrino flavors were found in the following years, and with the discovery of the muon neutrino in 1962 \sidecite{PhysRevLett.9.36} and the tau neutrino in 2001 \sidecite{KODAMA2001218}, the current theory of neutrinos in the standard model (SM) was established.

Although neutrinos were first believed to be massless, experimental evidence showing the existence of mixed neutrino states started to appear in the 1960s \sidecite{homestake}.
Mixing between different physical representations of neutrinos is proof for differences in their masses.
The resulting phenomenon of neutrino oscillations can be incorporated into the standard model by extending it to include massive neutrinos.
How massive they are and how strong is the mixing between neutrino states has to be obtained from measurement.
Today there are a variety of precision oscillation experiments using solar, reactor and atmospheric neutrinos to tighten the constraints on the neutrino oscillation parameters.
IceCube is one of those leading experiments probing the oscillation theory with atmospheric neutrinos.

The IceCube Neutrino Observatory \sidecite{2017JInst..12P3012A_Instrumentation_Systems} was constructed between 2004 and 2010 at the geographic South Pole.
It is the first cubic kilometer Cherenkov neutrino detector and consists of 5160 optical sensors attached to 86 strings, drilled down to a maximum depth of $\sim2500$\,m into the Antarctic ice.
Neutrinos are detected by the Cherenkov light that is emitted by secondary particles produced in neutrino-nucleon scattering interactions in the ice.
With DeepCore, a more densely instrumented sub-array of IceCube, the neutrino detection energy threshold can be lowered to approximately 5\,GeV.

At these energies, the similarity in event signatures poses difficulties in identifying different neutrino flavor interactions.
Muon neutrino charged-current interactions produce light tracks as opposed to charged-current interactions of electron and tau neutrinos as well as neutral-current interactions of all neutrinos that produce light cascades.
The sparse instrumentation of IceCube makes it more challenging to separate track- and cascade-like events.
In this thesis, a novel method to distinguish those two event types is developed.
In contrast to previously used univariate separation techniques, the multivariate machine learning method applied here maximizes the use of information from the detector response.
Through the use of a Gradient Tree Boosting algorithm the separation of events in track and cascade is improved.
As a result of the improved separation, the uncertainty to the atmospheric neutrino oscillation parameters $\Delta \mathrm{m}^{2}_{32}$ and $\theta_{23}$ is significantly reduced.

This thesis is structured as follows. 

% Chapter~\ref{chap:theoretical_background} starts with an introduction to neutrino properties and their SM interactions followed by the description of atmospheric neutrinos and the formulation of standard oscillations.
% Chapter~\ref{chap:neutrino_detection} introduces the IceCube Neutrino Observatory, its detection principle along with the observed event signatures and the reconstruction method used for this work.
% Chapter~\ref{chap:improving_pid} describes the machine learning technique that was utilized to develop a novel particle identification variable.
% Chapter~\ref{chap:oscillation_parameter_measurement} applies this method to achieve improved sensitivities in the measurement of atmospheric neutrino oscillations.
% Finally, Chapter~\ref{chap:summary} gives a concluding summary and a short outlook.
