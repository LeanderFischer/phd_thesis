\setchapterstyle{kao}
\setchapterpreamble[u]{\margintoc}

\chapter{Introduction}
\labch{intro}

The neutrino was postulated in 1930 by Wolfgang Pauli \sidecite{Pauli:1930pc}, in order to explain the observed continuous energy spectrum of electrons originating from beta decays. In 1956, Cowan and Reines confirmed this prediction of a neutral, light particle through the measurement of the electron neutrino via the inverse beta decay \sidecite{Cowan103}. With the discovery of the muon neutrino in 1962 \sidecite{PhysRevLett.9.36}, and the observation of tau neutrino interactions in 2001 \sidecite{KODAMA2001218}, the current three-flavor neutrino picture of the standard model was established. In this description, neutrinos are purely left-handed chiral particles that only interact via the weak force and are predicted to be massless.

However, experimental observations of neutrino flavor transitions started to appear in the 1960s \sidecite{homestake}, which can only be explained by the mixing between different neutrino states and the existence of non-zero mass differences. This means that at least two of the three neutrinos must have a non-zero mass. Extending the standard model, to describe the three neutrino flavor eigenstates as a superposition of three neutrino mass eigenstates, allows the description of the observed phenomenon of neutrino oscillations. To date, a variety of experiments are observing neutrino oscillations and have measured the neutrino mixing parameters and the mass differences with high precision. But the origin of the neutrino masses remains unknown.

A possible resolution to this problem is the existence of right-handed neutrinos and additional heavy mass states. Through a small mixing with the standard model neutrinos, these could explain the observed neutrino masses and their smallness. With masses $\gg$\si{\electronvolt} they are called heavy neutral leptons and could also play an important role in explaining further problems such as the baryon asymmetry of the universe or serve as dark matter candidates. While they are almost sterile, the small mixing to the standard model neutrinos allows them to participate in weak interactions, which makes experimental searches for these particles possible.

This work presents the first search for heavy neutral leptons using atmospheric neutrinos. The data is collected by the IceCube Neutrino Observatory, which is a cubic kilometer Cherenkov neutrino detector that was constructed between 2004 and 2010 at the geographic South Pole \sidecite{2017JInst..12P3012A_Instrumentation_Systems}. It consists of 5160 optical sensors attached to 86 strings, drilled vertically into the Antarctic glacial ice to a maximum depth of $\sim$\SI{2500}{\meter}. Neutrinos are detected via the Cherenkov light that is emitted by secondary particles produced in neutrino-nucleon scattering interactions in the ice.

For this search, the standard three flavor neutrino model is extended by adding a fourth GeV-scale mass state and allowing mixing with the tau neutrino through the mixing parameter \ut4. The strength of this mixing is tested using atmospheric neutrinos as a source flux. Muon neutrinos that oscillated into tau neutrinos can produce heavy neutral leptons through neutral current interactions, which then decay back to standard model particles. Both production and decay may deposit light in the detector, leading to a unique signature of two cascades at low energies.

A thorough investigation of this unique low-energy double cascade signature of heavy neutral leptons in IceCube is performed. A benchmark reconstruction performance is estimated using a well established IceCube reconstruction tool, after optimizing it for low-energy double cascade events. The limitations of the detector to observe these events are identified, and their origins are discussed. Since identifying the low-energy double cascade signature proved to be very challenging, an analysis is performed by searching for the shape imprint of events from heavy neutral leptons on top of the standard model neutrino sample. The measurement is performed through a binned, maximum likelihood fit, comparing the observed data to the expected events from atmospheric neutrinos and heavy neutral leptons. Three discrete heavy neutral lepton mass values, $m_4$, of \SI{0.3}{\gev}, \SI{0.6}{\gev}, and \SI{1.0}{\gev} are tested using ten years of data, collected between 2011 and 2021. The fits constrain the mixing parameter to \ut4$ < 0.19\;(m_4 = \SI{0.3}{\gev})$, \ut4$ < 0.36\;(m_4 = \SI{0.6}{\gev})$, and \ut4$ < 0.40\;(m_4 = \SI{1.0}{\gev})$ at \SI{90}{\percent} confidence level. No significant signal of heavy neutral leptons is observed for any of the tested masses, and the best fit mixing values obtained are consistent with the null hypothesis of no mixing. This first analysis lays the fundamental groundwork for future searches for heavy neutral leptons in IceCube.

The author was also involved in several projects, which are not directly related to the main analysis presented in this thesis. In close collaboration with a former colleague (Alex Trettin), a novel method to treat detector uncertainty effects in IceCube was developed. It was documented in a few author paper, and is now one of the default methods to incorporate detector uncertainties in atmospheric neutrino analyses in IceCube. This method will also be used in the main analysis of this thesis. Throughout the last years, the author was strongly involved in updating and maintaining the open source analysis framework PISA, which is used in many analyses using the IceCube atmospheric neutrino samples.

This thesis is structured as follows. After introducing the standard model and the extensions to include heavy neutral leptons in \refch{sm_neutrinos_properties}, the IceCube Neutrino Observatory and its detection principle are described in \refch{icecube}. \refch{signal_simulation} explains two avenues to simulate the heavy neutral lepton signal in IceCube. A model-independent event generator was developed exclusively by the author, to benchmark the reconstruction performance and cross-check the model-dependent simulation. For the latter, a skeletal structure was constructed by collaborators, before the author took over and implemented the full model-dependent simulation chain, continuously optimizing and testing it, before producing and processing the full samples for the main analysis. Both the study of the performance of IceCube to reconstruct and identify heavy neutral lepton events presented in \refch{simulation_and_processing}, and the main analysis discussed in \refch{shape_analysis} were developed and performed independently and are original work of the author.
