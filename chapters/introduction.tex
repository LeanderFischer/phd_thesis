\setchapterstyle{kao}
\setchapterpreamble[u]{\margintoc}

\chapter{Introduction}
\labch{intro}

The neutrino was postulated in 1930 by Wolfgang Pauli~\sidecite{Pauli:1930pc}, in order to explain the observed continuous energy spectrum of electrons originating from beta decays. In 1956, Cowan and Reines confirmed this prediction of a neutral, light particle through the measurement of the electron neutrino via the inverse beta decay~\sidecite{Cowan103}. With the discovery of the muon neutrino in 1962~\sidecite{PhysRevLett.9.36}, and the observation of tau neutrino interactions in 2001~\sidecite{KODAMA2001218}, the current three-flavor neutrino picture of the standard model was established. In this description, neutrinos are purely left-handed chiral particles that only interact via the weak force and are predicted to be massless.

However, experimental observations of neutrino flavor transitions started to appear in the 1960s~\sidecite{homestake}, which can only be explained by the mixing between different neutrino states and the existence of non-zero mass differences. This means that at least two of the three neutrinos must have a non-zero mass. Extending the standard model to describe the three neutrino flavor eigenstates as a superposition of three neutrino mass eigenstates, allows the description of the observed phenomenon of neutrino oscillations. To date, a variety of experiments are observing neutrino oscillations and have measured the neutrino mixing parameters and the mass differences with high precision. But the origin of the neutrino masses remains unknown.

A possible resolution to this problem is the existence of right-handed neutrinos and additional heavy mass states. Through a small mixing with the standard model neutrinos, these could explain the observed neutrino masses and their small magnitude. With masses $\gg$\si{\electronvolt} they are called heavy neutral leptons and could also play an important role in explaining further problems such as the baryon asymmetry of the universe or serve as dark matter candidates. Through their small mixing to standard model neutrinos, experimental searches for these particles are possible.

This work presents the first direct search for heavy neutral leptons using atmospheric tau neutrinos. The data is collected by the IceCube Neutrino Observatory, which is a cubic-kilometer Cherenkov neutrino detector that was constructed between 2006 and 2010 at the geographic South Pole~\sidecite{2017JInst..12P3012A_Instrumentation_Systems}. It consists of 5160 optical sensors attached to 86 strings, drilled vertically into the Antarctic glacial ice to a maximum depth of $\sim$\SI{2500}{\meter}. Neutrinos are detected via the Cherenkov light that is emitted by secondary particles produced in neutrino-nucleon scattering interactions in the ice.

For this search, the standard three flavor neutrino model is extended by adding a fourth GeV-scale mass state and allowing mixing with the third lepton generation through the mixing parameter \ut4. The strength of this mixing is tested using atmospheric neutrinos as a source flux. Muon neutrinos that oscillate into tau neutrinos would produce heavy neutral leptons through neutral current interactions, which then decay back to standard model particles. Both production and decay may deposit light in the detector, leading to a unique signature of two cascades at low energies.

A thorough investigation of this unique low-energy double-cascade signature of heavy neutral leptons in IceCube is performed. A benchmark reconstruction performance is estimated using a well established IceCube reconstruction tool, after optimizing it for low-energy double-cascade events. The limitations of the detector to observe these events are identified, and their origins are discussed. Since identifying the low-energy double-cascade signature proved to be very challenging, an analysis is performed by searching for the shape imprint of events from heavy neutral leptons on top of the standard model neutrino sample. The measurement is performed through a binned, maximum likelihood fit, comparing the observed data to the expected events from atmospheric neutrinos and heavy neutral leptons. Three discrete heavy neutral lepton mass values, $m_4$, of \SI{0.3}{\gev}, \SI{0.6}{\gev}, and \SI{1.0}{\gev} are tested using ten years of data, collected between 2011 and 2021. No significant signal of heavy neutral leptons is observed for any of the tested masses, and the best fit mixing values obtained are consistent with the null hypothesis of no mixing. The fits constrain the mixing parameter to \ut4$ < 0.19\;(m_4 = \SI{0.3}{\gev})$, \ut4$ < 0.36\;(m_4 = \SI{0.6}{\gev})$, and \ut4$ < 0.40\;(m_4 = \SI{1.0}{\gev})$ at \SI{90}{\percent} confidence level. This first analysis lays the fundamental groundwork for future searches for heavy neutral leptons in IceCube.

This thesis is structured as follows. After introducing the standard model and the extensions to include heavy neutral leptons in \refch{sm_neutrinos_properties}, the IceCube Neutrino Observatory and its detection principle are described in \refch{icecube}. Two avenues to simulate the heavy neutral lepton signal in IceCube are presented in \refch{signal_simulation}. A model-independent event generator was developed to benchmark the reconstruction performance and cross-check the model-dependent simulation. The study of the performance of IceCube to reconstruct and identify heavy neutral lepton events is presented in \refch{simulation_and_processing}, and the main analysis is discussed in \refch{analysis}.

The author played a leading role in the development of both the model-dependent and independent heavy neutral lepton simulations described in \refch{signal_simulation}. This includes the calculation of heavy neutral lepton cross sections and possible decay modes, as well as general simulation production and validation. The optimization and benchmarking of the low-energy double-cascade reconstruction was performed exclusively by the author, as was the search for heavy neutral leptons using 10 years of atmospheric neutrino data. Crucial for this analysis was an accurate assessment of the impact of systematic uncertainties, and for this purpose the author contributed significantly to the development of the new method described in \refsec{ultrasurfaces} and~\sidecite{Fischer_2023}. In addition, the author was a primary maintainer of the open source analysis framework~\cite{pisa_software}, which was used to perform the analysis. Finally, the author also contributed to the development of multi-PMT digital optical modules for the IceCube Upgrade project by characterizing the analog front-end electronics with early prototypes. This hardware work is not described in the thesis.
