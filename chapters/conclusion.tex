\setchapterstyle{kao}
\setchapterpreamble[u]{\margintoc}

\chapter{Conclusion}
\labch{conclusion}

The standard model description of neutrinos as three flavors of purely left-handed chiral particles fails to explain the observed experimental evidence of their existing, non-zero masses. Extending the model to include right-handed neutrinos and additional heavy mass states is one viable explanation to this dilemma. A small mixing between the standard model neutrinos and the additional heavy mass states could explain the observed neutrino masses and their smallness. With large masses ($\gg$\si{\electronvolt}) these states are called heavy neutral leptons and the small mixing with the standard model particles allows them to participate in weak interactions, which makes experimental searches for these particles possible.

In this work, the first direct search for GeV-scale heavy neutral leptons using atmospheric tau neutrinos is presented. The standard three flavor neutrino model is extended by a fourth mass state, allowing mixing with the third lepton generation via a non-zero mixing parameter \ut4. The strength of this mixing is measured using ten years of IceCube DeepCore data, by performing a binned, maximum likelihood fit, comparing the observed data to the expected events from atmospheric neutrinos and heavy neutral leptons. Three discrete mass values, $m_4$, of \SI{0.3}{\gev}, \SI{0.6}{\gev}, and \SI{1.0}{\gev} are tested and no significant signal of heavy neutral leptons is observed. The fits constrain the mixing parameter to \ut4$ < 0.19\;(m_4 = \SI{0.3}{\gev})$, \ut4$ < 0.36\;(m_4 = \SI{0.6}{\gev})$, and \ut4$ < 0.40\;(m_4 = \SI{1.0}{\gev})$ at \SI{90}{\percent} confidence level.

Additionally, the potential of IceCube to observe the unique signatures of heavy neutral leptons is investigated. If both the production and the subsequent decay of the heavy neutral lepton happens inside the detector, a double-cascade signature is expected at low energies. After optimizing a double-cascade reconstruction for low-energy events inside the DeepCore volume, the performance of the reconstruction is investigated in detail. This revealed a number of difficulties, where the most challenging difficulty is the low light depositions of these events. It can be concluded that the main reason for the bad reconstruction performance is the low-energy of the second cascade.

Both the first search for heavy neutral leptons and the investigation of the potential to observe their unique signatures lays the fundamental groundwork for future searches for heavy neutral leptons in IceCube. Despite the observed limits being several orders of magnitude below the current leading limits on \ut4, this initial result serves as a proof of concept for HNL searches using atmospheric neutrinos. The analysis is expected to improve in the future through more sophisticated reconstruction techniques and a targeted event selection. Additionally, the future IceCube Upgrade detector, which is expected to start data taking in 2026, will significantly enhance the light detection of low-energy cascades and should yield a better chance to identify the unique HNL signature.
