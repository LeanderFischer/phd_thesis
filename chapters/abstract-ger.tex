Die Beobachtung von Neutrino-Oszillationen hat gezeigt, dass Neutrinos eine von Null verschiedene Masse haben. Dieses Phänomen wird nicht durch das Standardmodell der Teilchenphysik beschrieben, aber eine mögliche Erklärung für dieses Dilemma ist die Existenz von schweren neutralen Leptonen in Form von rechtshändigen Neutrinos. Abhängig von ihrer Masse und Kopplung zu den Neutrinos des Standardmodells könnten diese Teilchen auch eine wichtige Rolle bei der Lösung weiterer unerklärter Beobachtungen wie Dunkler Materie und der Baryonenasymmetrie des Universums spielen. Diese Arbeit präsentiert die erste Suche nach schweren neutralen Leptonen mit dem IceCube Neutrino-Observatorium. Das standardmäßige Drei-Flavor-Neutrino-Modell wird erweitert, indem ein vierter Massenzustand im GeV-Bereich hinzugefügt wird und eine Mischung mit dem Tau-Neutrino durch den Parameter \ut4 erlaubt wird. Es werden drei Massenwerte für schwere neutrale Lepton, $m_4$, von \SI{0.3}{\gev}, \SI{0.6}{\gev} und \SI{1.0}{\gev} getestet, wobei zehn Jahre Daten von den Jahren 2011 bis 2021 verwendet werden. Für keine der drei getesteten Massen wird ein signifikantes Signal von schweren neutralen Leptonen gemessen. Die resultierenden Einschränkungen für den Mischungsparameter sind \ut4$ < 0.19\;(m_4 = \SI{0.3}{\gev})$, \ut4$ < 0.36\;(m_4 = \SI{0.6}{\gev})$ und \ut4$ < 0.40\;(m_4 = \SI{1.0}{\gev})$ im \SI{90}{\percent}-Konfidenzniveau. Diese erste Analyse legt die grundlegende Basis für zukünftige Suchen nach schweren neutralen Leptonen in IceCube.
