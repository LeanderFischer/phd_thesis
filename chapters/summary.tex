\setchapterstyle{kao}
\setchapterpreamble[u]{\margintoc}

\chapter{Summary and Outlook}
\labch{summary_and_outlook}

what was done?

1. set up model dependent and independent signal simulation for low energy double cascade events from HNL production and decay inside IceCube DeepCore

2. estimate performance of reconstructing and identifying these events

3. search for (cascade-like) events in 10 years of IceCube DeepCore data


\section{Performance to Detecting Low Energetic Double Cascades}

 


From the investigation of the good and badly reconstructed events, it can be concluded that the main reason for the bad reconstruction is the low energy of the second cascade. Despite the fact that the split into the two populations was very rudimentary, it is clear that this is the dominant cause, while other factors, like the position of the second cascade, or the potentially bad input seed direction are also contributing. For a thorough investigation, a more sophisticated separation would be needed.


\section{Search for Heavy Neutral Lepton Production and Decay}

\subsection{Agreement with Standard Model Three-Flavor Oscillation Measurement}

\todo{Compare here the best fit oscillation parameters to the FLERCNN results and try to quantify it, stating the pitfalls of the comparitions (satistically fully dependent) (RED)}

\subsection{Comparison to Other Experiments}

\todo{make summary plot (masses and mixing limits on one) and then discuss wrt to other experiments? (RED)}

\section{Outlook}

\subsection{Shape Analysis Improvements}

\begin{itemize}
    \item estimate full contribution from cascade only events (underestimated due to limited sampling distributions)
    \item include double cascade classifier into Binning
    \item further optimize binning
\end{itemize}

\subsection{Test Coupling to Electron/Muon Flavor}

\subsection{Test Additional Coupling Processes}

\subsection{IceCube Upgrade}
