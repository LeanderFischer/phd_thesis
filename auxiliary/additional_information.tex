
\thispagestyle{plain}
\begin{minipage}[c][0.4\textheight][b]{0.9\textwidth}
    \huge
    \textbf{Foreword}
    \normalsize
    \par
    \vspace{1.0cm}
    Before diving into the scientific content of my work, I would like to give some editorial remarks to smoothen the reading experience. Throughout the thesis, acronyms and experiment names are introduced in \textit{italic font}, the first time they are mentioned, but are used in normal font from then on. The same goes for software packages, which are initially mentioned in \textsc{small caps font}. One of the key features of the kaobok template - the big margin - is put to good use to house tables, figures, and additional notes, but also to highlight selected references. Of course, all references are listed in their full extent in the bibliography at the end, but additionally, some (but not necessarily all) of them will be highlighted in the margin next to where they appear to allow for an uninterrupted flow of reading.
\end{minipage}
