\chapter*{Foreword}

Before diving into the scientific content of my work, I would like to give some editorial remarks to smoothen the reading experience. Throughout the thesis, acronyms are introduced in \textit{italic font}, the first time they are mentioned, but are used in normal font from then on. The same goes for software packages, which are initially introduced in \textsc{small cap font}. One of the key features of the kaobok template - the big margin - is put to good use to house tables, figures, and additional notes, but also to highlight selected references. All references are listed in their full extent in the bibliography at the end, but additionally, some (but not necessarily all) of them will be highlighted in the margin next to where they appear to allow for an uninterrupted flow of reading.

During my time at desy and in IceCube, I have been involved in several projects, which are not all directly related to the main analysis presented in this thesis. I will give a brief overview of my scientific contributions and how they are related to the main analysis.

In close collaboration with a former colleague (Alex Trettin), we developed a novel method to treat detector uncertainty effects in IceCube, which we documented in a few author paper and which is now the default method to treat detector uncertainties in atmospheric neutrino analyses in IceCube.


\textbf{Work related (what is my original work):}
\begin{itemize}
    \item developed a method to treat detector uncertainty effects in close collaboration with former colleague (A. Trettin) which is well documented in a published paper and her thesis but will be introduced in \refsec{ultrasurfaces} and used in the main analysis of this work
    \item the model independent simulation chain described in \refsec{model_independent_simulation} was developed exclusively by myself
    \item for the model dependent generator presented in \refsec{model_dependent_simulation}, the skeletal structure was constructed by collaborators, before I took over and implemented the full model dependent simulation chain, including the correct decay widths calculations, custom cross-section, and the weighting scheme, continuously optimizing and testing it, before producing and processing the full samples for the main analysis
    \item say something about my PISA contribution
    \item both the study on how well IceCube can detect low energy double cascades in \refch{double_cascade_performance} and the main analysis in \refch{shape_analysis} were developed and performed by myself independently and are original work
\end{itemize}


% \textbf{Editorial remarks:}
% \begin{itemize}
%     \item Acronyms are introduced in \textit{italic font} the first time they are mentioned and will be used in normal font from then on.
%     \item Software packages will be introduced in \textsc{small cap font} and later used in normal font. 
%     \item All references are listed in its full extent in the bibliography, but additionally, a selection (but not necessarily all) of them will be highlighted in the margin next to where they appear to smoothen the flow of reading.
% \end{itemize}
