\thispagestyle{plain}
\begin{center}
	\vspace*{1cm}

	\LARGE
	\textbf{Search for Heavy Neutral Lepton Production and Decay with the IceCube Neutrino Observatory}
	\large

	\vspace{0.8cm}

	\textbf{Dissertation}\\
	zur Erlangung des akademischen Grades\\
	doctor rerum naturalium \\
	(Dr. rer. nat.) \\

	\vspace{0.5cm}

	im Fach: Physik \\
	Spezialisierung: Experimentalphysik\\

	\vspace{0.5cm}

	eingereicht an der \\
	Mathematisch-Naturwissenschaftlichen Fakultät\\
	der Humboldt-Universität zu Berlin\\

	\vspace{0.5cm}

	von\\
	\textbf{Leander Fischer M. Sc.}\\
	%	\vspace{0.8cm}
	geboren am 24. Oktober 1992\\
	in Heidelberg

	\vspace{0.5cm}

	Präsidentin der Humboldt-Universität zu Berlin\\
	Prof. Dr.-Ing. Dr. Sabine Kunst\\

	\vspace{0.5cm}

	Dekan der Mathematisch-Naturwissenschaftlichen Fakultät\\
	Prof. Dr. Elmar Kulke\\
\end{center}

\newpage
\thispagestyle{plain}

\begin{flushleft}	
	\vspace*{8cm}
	
	\textbf{No copyright} \\
	\cczero\ This book is released into the public domain using the CC0 code. To the extent possible under law, I waive all copyright and related or neighbouring rights to this work.
	
	To view a copy of the CC0 code, visit: \\\url{http://creativecommons.org/publicdomain/zero/1.0/}

	\medskip
	
	\textbf{Colophon} \\
	This document was typeset with the help of \href{https://sourceforge.net/projects/koma-script/}{\KOMAScript} and \href{https://www.latex-project.org/}{\LaTeX} using the open-source \href{https://github.com/fmarotta/kaobook/}{kaobook} template class.\\
	
	% \todo[inline, noinlinepar]{Link to code}
	% The source code of this thesis is available at:\\\url{xx}, \\while the scripts used to generate the

	\medskip

	% \textbf{Publisher} \\
	% First printed in Nov 2022 by Humboldt Universität zu Berlin
	% \todo[inline, noinlinepar]{add when first printed and by whom}
\end{flushleft}
	

% \newpage
% \thispagestyle{plain}

% \vspace*{5.0cm}

% \large
% \noindent A neutrino is not a big thing to be hit by. In fact it's hard to think of anything much smaller by which one could reasonably hope to be hit. And it's not as if being hit by neutrinos was in itself a particularly unusual event for something the size of the Earth. Far from it. It would be an unusual nanosecond in which the Earth was not hit by several billion passing neutrinos.\\

% \flushright --\textit{The Hitchhiker's Guide to The Galaxy}
